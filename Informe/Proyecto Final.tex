\documentclass[journal,trans]{IEEEtran}
\usepackage[spanish]{babel}
\usepackage[utf8]{inputenc}
\usepackage{graphicx}
\usepackage{amsmath}
\usepackage{amssymb}
\usepackage{multirow}
\usepackage[dvipsnames]{xcolor}
\usepackage{ifpdf}
\usepackage{cite}
\usepackage{url}
\hyphenation{op-tical net-works semi-conduc-tor}

\begin{document}

	\newcommand{\titlepaper}{Proyecto Final}
	
	\title{\titlepaper}
	
	\renewcommand\IEEEkeywordsname{Palabras clave}
	
	\author{\IEEEauthorblockN{Fiorella Delgado León, Jonathan Guzmán Araya}
	\IEEEauthorblockA{\\fiorelladelgado53@gmail.com, jonathana1196@gmail.com}
	\IEEEauthorblockA{\\Área Académica de Ingeniería en Computadores}
	\IEEEauthorblockA{\\Instituto Tecnológico de Costa Rica, Cartago, Costa Rica, 2021}
	}
	
	\markboth{Taller de Diseño Digital, \titlepaper}%
	{Shell \MakeLowercase{\textit{et al.}}: Bare Demo of IEEEtran.cls for Journals}
	
	% Note that keywords are not normally used for peerreview papers.
	
	\IEEEtitleabstractindextext{%
	\begin{abstract}
			Este documento presenta los resultados del proyecto final del curso.
	\end{abstract}
	\begin{IEEEkeywords}
		FPGA
	\end{IEEEkeywords}}
	
	% make the title area
	\maketitle
	
	\IEEEdisplaynontitleabstractindextext
	
	\IEEEpeerreviewmaketitle
	
	\section{Introducción}
	
	El procesador ARM es un componente clave de muchos sistemas integrados. El primer ARM se introdujo en 1985, bajo el nombre de Acorn RISC Machine (luego renombrado Advanced RISC Machine). La relativa simplicidad de los procesadores ARM los hizo adecuados para aplicaciones de bajo consumo permitiendo así su actual amplia adopción. ARM es inusual en el sentido de que no vende proesadores directamente, si no que autoriza a otras compañías a construir sus diseños de procesadores, a menudo como parte de un sistema en chip más grande, por ejemplo: Samsung, Altera, Apple, Qualcomm construyen procesadores ARM, ya sea utilizando microarquitecturas compradas en ARM o microarquitecturas desarrolladas bajo licencia ARM \cite{Gomar2018}.
	
	Empezamos el documento \cite{SarahLHarris2010} probando la lista de referencias.
	
	\section{Sistema Desarrollado}
	
	\section{Resultados}
	
	\section{Analisis de Resultados}
	
	\section{Conclusiones}
	
	\section{Bibliografía}
	
	\bibliographystyle{IEEEtran}
	\bibliography{myref}
	
\end{document}